\documentclass{article}
% \usepackage{graphicx}
\usepackage{url}

\begin{document}

\title{Social Density Estimation -- Experimental Setup}
\author{Guntur Dharma Putra}

\maketitle

% \begin{abstract}
% The abstract text goes here.
% \end{abstract}

%%%%%%%%%%%%%%%%%%%%%%%%%%%%%%%% 
% [Why How]
% Literature on Topic
% Literature on Method
% Theoretical Approach
% Find a Hole
% Look for debates

\section{Introduction} % (fold)
\label{sec:introduction}
This document presents an experimental setup for Social density estimation using WiFi. The correlation between the number of unique device (sensed using WiFi probe-request) and available Access Point in a particular area is investigated. This document also describes the experimental setup for MAC address randomization experiment. Furthermore, availability of voice activity will also take into account.

\section{Experimental Setup} % (fold)
\label{sec:experimental_setup}
The experimental setup is divided into two sections: WiFi sensing and MAC address randomization.

\subsection{WiFi Sensing} % (fold)
\label{sub:wifi_sensing}
The objective of WiFi sensing is to capture WiFi probe-request packets and scan and count available Access Point. As a validation, Voice Activity Detection (VAD) technique is also implemented.

\subsubsection*{Sensing setup} % (fold)
The experimenter will basically capture the probe-request packets, scan available Access Point, and detect for voice activity level. Several setups are as follows:

\label{ssub:sensing_setup}
\begin{description}
	\item[Static] The experimenter will remain still at a certain location. The instrument for this setup is:
	\begin{itemize}
		\item Laptop (with built-in microphone and network card)
	\end{itemize}

	\item[Dynamic] The experimenter will walk through the crowd. The instruments are:
	\begin{itemize}
		\item Raspberry Pi (with WiFi dongle and battery)
		\item smartphone (with built-in microphone)
	\end{itemize}
\end{description}

For dynamic sensing, the experimenter will try to put the instruments in a bag and carry the device.
% subsubsection sensing_setup (end)

\subsubsection*{Sensing Area} % (fold)
\label{ssub:sensing_area}
The area is classified to:
\begin{itemize}
	\item Outdoor\footnote{The density of crowd is determined by demographical data from the government.}
	\begin{itemize}
		\item High density of crowd (Vismarkt, Grotemarkt)
		\item Medium density of crowd (Vismarkt, Grotemarkt)
		\item Low density of crowd (Centrum in other village)
	\end{itemize}
	
	\item Indoor
	\begin{itemize}
		\item Large-sized hall (Study room bernoulliborg, deuisenberg, university library)
		\item Small-sized hall (McD westerhaven)
	\end{itemize}
\end{itemize}

\noindent\textbf{Note:}\\
University complex will also considered as it has medium to high crowd density but only (possibly) one available Access Point (\texttt{eduroam}). For each sensing, we record the GPS coordinates. All WiFi channels will be measured.
% subsubsection sensing_area (end)

\subsubsection*{Duration of Sensing} % (fold)
\label{ssub:duration_of_sensing}
The duration of sensing is classified to:
\begin{itemize}
	\item Short (15 to 20 minutes)
	\item Medium (30 to 40 minutes)
	\item Long (60 to 90 minutes)
\end{itemize}
% subsubsection time_and_duration_of_sensing (end)
% subsection wifi_sensing (end)

% \subsubsection*{Ground Truth Picture} % (fold)
% \label{ssub:ground_truth_picture}
% When running the experiment, please also take pictures, in four direction or panoramic. These pictures will be our ground truth.
% % subsubsection ground_truth_picture (end)

\subsection{MAC Address randomization} % (fold)
\label{sub:mac_address_randomization}
This research is trying to find a way to overcome MAC address randomization in Android. In order to do so, the experimenter will use laptop and scan for any MAC address changes in a location where no other WiFi probe-requests can be captured, i.e., remote areas.
% subsection mac_address_randomization (end)

\section{Data Processing and Analysis} % (fold)
\label{sec:data_processing_and_analysis}
Organizationally Unique Identifier\footnote{\url{https://en.wikipedia.org/wiki/Organizationally_unique_identifier}}

\subsection{Data Preprocessing} % (fold)
\label{sub:data_preprocessing}

% subsection data_preprocessing (end)

\subsection{Voice Activity Detection} % (fold)
\label{sub:voice_activity_detection}
In this analysis, we use library from bla bla.

The voice activity is classified in several level (dB).
% subsection voice_activity_detection (end)

\subsection{Data Visualization} % (fold)
\label{sub:data_visualization}

% subsection data_visualization (end)

We are also going to present some graphs: .
	- give graph of MAC address manufacturer.
	- comparison between devices (Laptop, smartphone, etc).
	- number of scanned probe per device.

- Filter based on RSSI. Find a literature that says the threshold of WiFi signal.
- To filter the data, i.e., to remove laptops or other device, use known MAC address filter.
- Take into account people going in and out from the scanned area.

% section data_processing_and_analysis (end)

% \bibliography{bibliography}{}
% \bibliographystyle{plain}
\end{document}
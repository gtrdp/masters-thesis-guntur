\documentclass{article}
% \usepackage{graphicx}

\begin{document}

\title{Social Density Estimation -- A Literature Review}
\author{Guntur Dharma Putra}

\maketitle

% \begin{abstract}
% The abstract text goes here.
% \end{abstract}

%%%%%%%%%%%%%%%%%%%%%%%%%%%%%%%% 
% [Why How]
% Literature on Topic
% Literature on Method
% Theoretical Approach
% Find a Hole
% Look for debates

\section{Introduction} % (fold)
\label{sec:introduction}
% What is crowd counting % Why and what are the benefits of crowd counting

% Video monitoring, which is widely adopted, has limitations such as weather conditions, illumination changes, limited viewing angle, density, and brightness problem. Furthermore, audio tone and RF signal based approach also have its own drawbacks. For example, although currently most people bring a smartphone with them, this does not imply that we could directly infer the level of social density in particular area since not everyone will turn on the WiFi module of their mobile phone, or even some people might have more than one WiFi enabled device.

% section introduction (end)

\section{Literature Review} % (fold)
\label{sec:literature_review}
This section describes some approaches of crowd counting, along with other implementation of RF signal sensing.

% GSM location has an issue with privacy\cite{thesis017}.
% MAC is only a proxy since it does not infer directly to personal information, such as name or contact.



% [Why How]
% Literature on Topic
% Literature on Method
% Theoretical Approach
% Find a Hole
% Look for debates

\section{Conclusion and Possible Research}
Write your conclusion here.

\bibliography{bibliography}{}
\bibliographystyle{plain}
\end{document}
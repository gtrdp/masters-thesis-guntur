\documentclass{article}
% \usepackage{graphicx}

\begin{document}

\title{Literature Review of Passive Behavioural Monitoring}
\author{Guntur Dharma Putra}

\maketitle

% \begin{abstract}
% The abstract text goes here.
% \end{abstract}

\section{Literature Review}
Some researches have been done with regard to estimating social density, especially more are focused on crowd density, with varieties in the aims. The aims are not only focused on behavioral monitoring, ranging from building mitigation to pilgrimage.

Video processing has limitations such as weather conditions, illumination changes, limited viewing angle, and density and brightness problem. 

GSM location has an issue with privacy\cite{thesis017}.

MAC is only a proxy since it does not infer directly to personal information, such as name or contact.

A research from \cite{thesis008} proposes a way to detect crowds using Bluetooth. The crowd density is quantized into 7 groups, ranging from nearly empty to extremely high (crowded). Several features were also devised in this research, ranging from bla bla. 
The method was chosen due to bla bla.
The experiments were set up for 3 times, with 4 hours of duration each. 10 students were recruited to carry out the experiments.
The results show that bla bla.

Furthermore, \cite{thesis014} alleges that the existence of social relationships is possible to be uncovered by using WiFi probe signals.

Human queue is also possible to be monitored using WiFi, as demonstrated in \cite{thesis012}. It is based on RSSI that is measured by a single WiFi monitor.

WiFi and Bluetooth were also used to estimate crowd densities and pedestrian flows in \cite{thesis011}.

A research \cite{thesis017} utilizes MAC address data to determine spatio-temporal movement of human in terms of space utilization.

Bluetooth data is also used to analyze spatio-temporal movements of visitors event in Belgium \cite{thesis016}.

Movements pattern and landmark preferences are possible to be extracted from publicly available photo repositories, such as Flicker and Panoramio, as presented in~\cite{thesis026}.

An interesting insight is found in~\cite{thesis031}, this research goal and method are really similar with our research.

A research~\cite{thesis030} is also a little bit similar with the Paul's research.

Bluetooth, again is proven to be a potential source of tracking socially contextual behavior, as seen in~\cite{thesis028}. Using Bluetooth trace, Chen, et. al. has shown the result with 85,8\% accuracy.

A combination of WiFi fingeprinting and PDR is used to monitor Indoor environment by means of crowdsourcing~\cite{thesis020}.

A work~\cite{thesis009} alleges that WiFi prevails Bluetooth in several criteria. Firsly, Bluetooth requires longer time to discover. More than 90\% of detected MAC address were WiFi MAC address. MAC is unique address for most IEEE 802 technologies.

An online survey was conducted in~\cite{thesis001} to collect sensor sources and contexts that are relevant for ambulatory assessment. This is important as existing mobile apps only provide time-triggered prompts for experience sampling method rather than event-triggers. In fact, event-trigger prompts could give more accurate information as it is sampled at the moment which is of interest for a psychologist. The result shows that most relevant sensors are time, date, and user activity. However, location, notifications, and accelerometer data is also of interest. However, several issue emerge, i.e., not all of the sensors are accessible in Android, as some of them require root privilege.

A work~\cite{thesis006} tried to count the crowd using CSI, which is proven to have a monotonic relation with the number of moving people. The result seems promising, although some errors are observed.

A more energy efficient method to exploit sensor in smartphone is presented in~\cite{thesis040}. It makes use of, what they called, \textit{Smartphone App Opportunities}. The approach is named Piggyback Crowd Sensing (PCS).

Bluetooth has again proven to be one reliable method to estimate crowd density~\cite{thesis041}. The work alleges that it could even reach 82\% accuracy in the best case.

\cite{thesis042}~describes the possibility to use ZigBee to estimate crowd density by measuring the RSSI and LQI. This approach requires prior infrastructure.

More approach on WSN is described in~\cite{thesis043}. With similar solution in~\cite{thesis042},~\cite{thesis043} employs more WSN. It has normal and large-scale experiment.

\cite{thesis022}~explains the possibility of tracking people movement and contact by using bluetooth and wifi.

Another point of view to track pedestrian flocks is presented in~\cite{thesis033}. It uses WiFi signals with 3 different features to infer the flocks.

A paper~\cite{thesis045} presents a method that combine geo-fencing with coarse WiFi localization for building evacuation.

An example of crowd monitoring is presented in~\cite{thesis050}, where it is implemented for Hajj in Mecca, Saudi Arabia. It utilizes RFID tags along with a specialized app for monitoring the pilgrims.

[Why How]
Literature on Topic
Literature on Method
Theoretical Approach
Find a Hole
Look for debates

\section{Conclusion}
Write your conclusion here.

\bibliography{bibliography}{}
\bibliographystyle{plain}
\end{document}
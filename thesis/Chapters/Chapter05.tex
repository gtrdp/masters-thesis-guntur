%!TEX root = ../thesis-guntur.tex
%************************************************
\chapter{Regression Model and Discussion}
\label{ch:model-and-discussion} % $\mathbb{ZNR}$
%************************************************
Discussion: here is where you discuss the results, where you are interpreting the outliers and overall trends in your figures. Here you can dare a bit more and come to conclusions that are compatible with the evidence you found, but not necessarily unique. 

Explain about the previous chapter, begin with a good starting point of a chapter.

Explain the limitations of the thesis. Where and how the method works best.

present the result of each location. and compare within days
present the result in grotemarkt, and explain the difference between time of scanning.

Explain that using classification might give better accuracy that regression.

Also give graph of MAC address manufacturer.

Explain using bar chard about the comaprison of maximum unique device and head count: average, in each day. Also apply this for timely related 

Use wigle to explain that this could not be generalized. Mention comparison of one place and another place. in other cities or contries.

Mention about VAD and explain why it is not applicable.

Explain that this method does not work in restricted environment, such as university location.

Looking for microphone recording in phone.
% http://electronics.stackexchange.com/questions/59157/over-what-frequency-range-can-the-microphone-of-smartphone-receive-the-sound
% http://www.makeuseof.com/tag/great-tips-recording-audio-smartphone-tablet/
% http://www.scienceprog.com/long-range-directional-microphones-myth-and-reality/
% http://www.epanorama.net/newepa/2014/09/08/sound-level-measuring-with-android-phone/
% seems legit http://www.analog.com/library/analogdialogue/archives/46-05/understanding_microphone_sensitivity.html?doc=an-1328.pdf

Possible research using dB (decibels) to identify social density.
Decibels does not correlate linearly, but rather logaritmically. -> altought it turned out that it is not really good, ie., it does not have a good correlation.

0.3 of correlation means that 30% of the variation is explained.

explain that the IE method is not really 100% correct. prove it by using LG nexus randomized and real probe request.

Argument, writing:
it is hard to count people inside the bus.
we rely on an assumption that says everyone brings their own mobilephone with them, thus we can track it, just like what retail companies do.
To avoid randomized mac, we use discretize the monitoring, instead of doing it continuously for a long time, we did it in separate scanning interval.
Why dont you use 2 devices for separate purpose? one for probe request and one for access point.
Limitation: This work is only limited to free neighbourhood, i.e., this does not apply to a wifi restricted location such as university buildings.

Important, writing: also address the microphone sensitivity.

Working on decibels.
% https://support.biamp.com/General/Audio/Peak_vs_RMS_Meters
Important, writing: also address the microphone sensitivity.
% https://support.biamp.com/General/Audio/Microphone_sensitivity
Sound level decreases by 6dB with each doubling of distance from the source.
The sound is already using ambient noise reduction.
Those microphones are attuned to a specific (and rather narrow) range of sound intensity.

cross validation is used to validate the mode, explain a little bit

quantitative, continuous result: regression
classes, discrete result: classification

Explain that I used R to do data analysis.
Also mention what packages are used for each model, as a summary, better to draw a table about all classes used.

Explain a little bit about 10 folds cross validation.

Mention eps-regression and nu-regression in SVM

sections:
- linear model
- non linear model
\section{Discussions} % (fold)
\label{sec:discussions}
Possible explanation: explain that the number of access point might increase in a crowd because of the ad hoc access point.
also use % https://wigle.net/ database of public WiFi
We expect that this affect the ap count, but it turned out that it does not.

Turns out that multiple APs are using the same eduroam as their SSID, however, they have different MAC addresses. -> explain as a fact.

When android is in energy saving mode, the OS prohibits any app for doing WiFi Scan or any other scan.

[Possible problem] A drawback of scanning WiFi probe request: people located in other room (not within the person's eyesight) is still detectable, thus, worsen the results.
Combine with sound
Combine with GPS location if there is no sound detected.

Counting people using wifi has an issue: it can detect people through the walls, which people don't see. what do you think?

Explain that this method does not work in restricted environment, such as university location. Writing: explain why we do not do that in school/university, because WiFi is highly restricted in campus, i.e., we are not allowed to have individual access point in school.

We also select daytime as the preferred scanning time because that is the time when most crowd are observable. This implies that our collected data resemble only daytime duration, meaning that conclusion might only be able to deduct for daytime.


Question: Does weather affect WiFi performance?

Explain the result in percentage as well, instead of manual count. -> for the error in the modeling.

the setting may vary in cities or even countries.

plot the graph of head count (real vs predicted) vs parameters (ap count, or anything else)

present the evaluation in people count and also percentage

why the some SN disappear.

people might bring their own Access Point.

say that low level of ambienet noise is highly possible of low level area: cite the graph, both in result and appendix

We are interested in seeing the effect of scanning time, ie, scanning in the morning or afternoon as crowd condition are really dynamic.

% conclusion
Explain by conclusion, not using whole data
effect of day
effect of location
effect of scanning time
put it in a whole picture
signal strength: WiFi and probe request

Those microphones are attuned to a specific (and rather narrow) range of sound intensity.

mention in the appendix, that we can see the crowd count almost tell us nothing.

why regression, not classification?

talk about the result of speaker count result. say that it is not accurate and reliable, thus we do not use that.

Although probe request based estimation is promising, some drawbacks are also present. In this method, we are not able to distinguish the type of devices, i.e., whether it is a smartphone, tablets, or computers. Although recently one usually brings a smartphone~\cite{thesis047}, which means we can deduce that a smartphone means a person present, there is also a possibility that one brings more devices or no devices at all.

Compared to the probe request based estimation, this (what this?) method could not detect people through walls or buildings and thus does not really represent the actual condition of the location. However, we consider this (what this) method to be closer to the ground truth.

% residual information
% http://repo.thehackademy.net/depot_madchat/racine/reseau/wireless/IDS/wlan-mac-spoof.pdf
% http://blogs.cisco.com/wireless/apple-ios-8-and-mac-randomization-what-it-means-for-ciscos-connected-mobile-experiences-cmx-solution
% http://community.arubanetworks.com/t5/AAA-NAC-Guest-Access-BYOD/iOS-8-amp-MAC-Address-Randomization/td-p/169900
% http://www.cso.com.au/article/547177/apple_randomises_mac_addresses_ios_8_killing_off_key_ad-tracking_tool/
% http://www.imore.com/closer-look-ios-8s-mac-randomization
% https://en.wikipedia.org/wiki/MAC_address

Crowd++ Using the code from the original version.
Remove unrelated code block, such as the relation with smartphone things: battery, calibration (because we do not use semisupervised learning based on the owner voice) etc.
The code also uses Java Speech Toolkit (JSTK) from Speech Group at Informatik 5, Univ. Erlangen-Nuremberg, GERMANY.
% https://github.com/sikoried/jstk
Most time spent on debugging the code.
Crowdpp also uses YIN pitch tracking algorithm:
% http://recherche.ircam.fr/equipes/pcm/cheveign/ps/2002_JASA_YIN_proof.pdf
Finally got the code to work, bug found in file format and feature extraction (delete the feature after completed, otherwise the next process will append it which will cause Exception)
Found SoX, an audio recorder (and even more) for sound recording:
% http://sox.sourceforge.net/sox.html
Explain the audio result: what is pklv and rms.

Explain the limitation.

%*****************************************
%*****************************************
%*****************************************
%*****************************************
%*****************************************

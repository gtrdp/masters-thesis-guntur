%!TEX root = ../thesis-guntur.tex
%************************************************
\chapter{Discussion}
\label{ch:discussion} % $\mathbb{ZNR}$
%************************************************
% In this chapter we look at to what extend these questions have been answered, what problems arose from answering them and what future directions this work can go in.
We discuss our findings of the experiments in the following topics, ambient noise recording, ground truth approximation, scanning time effect, sensor readings correlations, the implications of the data modeling, and the limitation of the present study.

\section{Ambient Noise Recording} % (fold)
\label{sec:ambient_noise_recording}
We recorded the ambient noise and extracted the peak-level (\ac{PKLV}) and root-mean-square (\ac{RMS}) of the recordings, which are highly correlated, with $\rho=0.77$ (see~\autoref{fig:scatterplot-matrix}). Initially, we expected to see a strong correlation of ambient noise and the level of social density, i.e., the location which has high level of social density also has high value of ambient noise, but the result says otherwise. We can see in the scatter plot matrix of the dataset, \autoref{fig:scatterplot-matrix}, that only the correlation of head count and peak level which is more than $0.5$ ($\rho=0.58$). The other correlations of social density levels (head count or device count) and ambient noise (peak level or root-mean-square) are below $0.5$.

In the scatter plot of~\autoref{fig:scatterplot-matrix}, we can see that some of the low social density values (less than 20) have high ambient noise value as well (more than -20dB), which means we also observed more noise in the area where less crowds were observable. However, we can also see in~\autoref{fig:scatterplot-matrix} that no high social density values (more than 50 for device count and 20 for head count) are below -30dB, which means that high social density areas have high amount of noise. We conclude that high values of ambient noise mostly indicate a high level of social density. The line charts showing the peak level and root-mean-square of the ambient noise of all locations (\autoref{fig:audio-result-day4}, \autoref{fig:audio-result-day1}, \autoref{fig:audio-result-day2}, and \autoref{fig:audio-result-day3}) also support our conclusion. The graphs show that more crowded locations (Grote Markt and Paddepoel) have higher peak level or root-mean-square value than (Home and remote area) less crowded locations although some overlaps exist.

Furthermore, microphone sensitivity also affects the result of ambient noise recording. We used a laptop's built-in microphone to record the ambient noise. This kind of microphones (and ones installed in smartphones) is attuned to a specific (and rather narrow) range of sound intensity.

% see the trend, maybe use more locations, more data

% https://support.biamp.com/General/Audio/Microphone_sensitivity
% http://electronics.stackexchange.com/questions/59157/over-what-frequency-range-can-the-microphone-of-smartphone-receive-the-sound
% http://www.makeuseof.com/tag/great-tips-recording-audio-smartphone-tablet/
% http://www.scienceprog.com/long-range-directional-microphones-myth-and-reality/
% http://www.epanorama.net/newepa/2014/09/08/sound-level-measuring-with-android-phone/
% seems legit http://www.analog.com/library/analogdialogue/archives/46-05/understanding_microphone_sensitivity.html?doc=an-1328.pdf


\section{Ground Truth Approximation} % (fold)
\label{sec:ground_truth_approximation}
We estimate the crowd count as a first approximation of the ground truth because it is known that getting the ground truth of crowd density in public spaces is difficult~\cite{thesis041}. We used time-lapse image based, which works by manually counting heads in the images, and WiFi's probe request based technique, works by counting unique \ac{MAC} addresses.
% Both methods rely on predefined assumptions, which say .

Although probe request based estimation is promising, some drawbacks are also present. In this method, we are not able to distinguish the type of devices, i.e., whether it is a smartphone, tablets, or computers. Although one mostly brings a smartphone in crowded public areas~\cite{thesis047}, which means we can deduce that a smartphone means a person present, there is also a possibility that one brings more devices or no devices at all.

Furthermore, the WiFi based technique is able to detect devices through walls, which is good or bad depending on the definition of social density. In indoor social density estimation, this is a bad approximation because WiFi might detect some people but in fact there are less or no people at all inside a particular room, as they are located in other rooms nearby. This is a potential threat, for instance, if our method detects some people but actually no one is present in the room. However, in outdoor social density estimation, WiFi based technique performs better than image based technique, as WiFi can see people through an obstacle, while image-based cannot. To sum up, we have to be careful in doing indoor monitoring. As an option, we can also combine the social density estimation with ambient recording to see whether there is a noise inside a room, as empty room is mostly quite.

Compared to the probe request based estimation, time-lapse image based technique cannot detect people through walls or buildings. This might be the reason why the image based technique detects less people than the WiFi based technique (see~\autoref{fig:total-population}). Furthermore, image based technique relies heavily on assumptions when vehicles, e.g., buses or cars, are captured in the image, as vehicles have very limited visual appearance of the people inside. We assume that there are a person in a car and five people in a bus. This assumptions may slightly bias the result.

\section{Scanning Time Effect} % (fold)
\label{sec:scanning_time_effect}
We tested the effect of scanning time to investigate whether scanning time can potentially affect the outcomes. Interesting findings about the scanning time are shown in~\autoref{fig:time-effect}. We see that the device count in each scanning time has different outcome, while the \ac{AP} count remains stable no matter when the scanning was performed. When we scanned the surroundings in the morning (09:00h), the device count was less than the \ac{AP} count, as there were not so many people present. However, the trend changed when we performed the scanning at 12:00h, as the device count surpassed the \ac{AP} count. We can also see an increasing trend at scanning time performed at 15:00h and decreasing trend at 18:00h.
If the trend continues, there might be lower device count than \ac{AP} count at night.
The findings presented in~\autoref{fig:time-effect} indicate that we also have to consider the scanning time.
% of the surroundings when the present study is implemented because different scanning time might have a very different outcome.

\section{Correlation Between Sensor Readings and Social Density} % (fold)
\label{sec:correlation_between_sensor_readings_and_social_density}
The parameter which has a strong correlation with head count or device count is the \ac{AP} count, as shown in the scatter plot matrix (\autoref{fig:scatterplot-matrix}). The correlation coefficients of \ac{AP} count vs device count, \ac{AP} count vs head count, and device count vs head count are 0.87, 0.85, and 0.86, respectively. We present in detail the correlation of head count, device count, and \ac{AP} count in~\autoref{sub:ap_and_social_density_correlation}.

As we can see in the scatter plots of the correlation of \ac{AP}, \ac{DC}, and \ac{HC}, presented in~\autoref{fig:ap-dc-scatterplot}, \autoref{fig:ap-hc-scatterplot}, and \autoref{fig:hc-dc-scatterplot}, the lower social density data, as at home or remote area, seems to be more concentrated than higher social density data, as at Paddepoel or Grote Markt, and the higher social density data is widely spread. However, this does not mean that the frequency of high social density values is bigger than the low social density value. In fact, we have more data for lower social density than higher social density, as shown in the histograms of \ac{AP}, \ac{DC}, and \ac{HC} in~\autoref{fig:scatterplot-matrix}.

If we look at~\autoref{fig:ap-hc-scatterplot}, \autoref{fig:ap-dc-scatterplot}, and \autoref{fig:hc-dc-scatterplot}, the data may also be used to perform a sort of localization, as each location shows different patterns. We can say that the \ac{AP} count is bound to an area, where each area has different social density levels. Thus, this method is somewhat telling where the smartphone actually is and what the average of social density level in that area is. On the other hand, we could infer where the location of the smartphone is using our proposed technique.
% although further study may be required to support this opinion.

% present the result of each location. and compare within days
In the beginning, we expect that the number of \ac{AP} follows the fluctuation of the number of people in a certain location, as people might bring their own portable WiFi transmitter to assemble an ad-hoc \ac{AP}, which will add up the number of \ac{AP} available in the location. However, this turned out to be a rare situation, as we do not see any trend between number of people and \ac{AP} count, as shown in line chart of the experiment presented in~\autoref{sec:line_charts}.

Moreover, if we look at the line chart, there is a fluctuation of \ac{AP} count, although actually available \ac{AP} count should be the same or stable across the time. This fluctuation might be caused by the instability of radio transmission that may affect the signal strength of the WiFi \ac{AP} and thus making some \ac{AP}s some not detected.

% Another interesting finding is that several \ac{AP}s are using the same \ac{SSID}, e.g., eduroam. This makes users only see one \ac{AP} available, while in fact, those \ac{AP} are using different \ac{MAC} address.

% \added{
\section{Social Density Prediction} % (fold)
\label{sec:social_density_prediction}
We have created several models to predict the social density using new data. The optimal model was achieved using \ac{k-NN} predictors, although it actually performs equally well compared to \ac{SVM} method. However, as opposed to linear model, \ac{k-NN} works better because \ac{k-NN} works by measuring the similarity of k-nearest neighbors of the predicted value, while linear model only fitting a straight line across the data. In fact, the correlation is not always perfectly linear.

The optimal model gives an residual error equals to 7.05 for head count and 13.14 for device count. The error means that, when the model is used to predict new head count data, the predicted value may be 7.05 higher or lower than the actual value, or 13.14 for new device count data. For instance, if our model predict 50 head count, it may be actually 57.05 or 42.95.
% implication on the future or the implementation in be happ

Moreover, compared to the value range of the head count or device count, which range from zero to 155 (head count) and zero to 288 (device count), the errors count 6.1\% for head count and 4.6\% for device count. In summary, the model is able to determine if the subject is currently in a considerably low or high social density area.
% device count: 0,04652777778
% head count: 0,06130434783
% }

\section{Limitations of the Present Study}
\label{sec:limitation_of_the_present_study}
The proposed method has several limitations.
% that limit its implementation.
% #1: location: range, indoor outdoor
The first limitation is the location. The proposed method only works at places where WiFi \ac{AP} use is not restricted, i.e., people are free to set up their own WiFi \ac{AP}. Some locations where the use of WiFi is restricted exist, for instance, University of Groningen complex, where eduroam is the only the available \ac{AP} and the occupants are not allowed to install their own \ac{AP}. If we collect data from this location, we will possibly get uncorrelated \ac{AP} count and social density.

Furthermore, our dataset consists of data collected from four different location. This dataset represents the situation of selected locations but possibly not for other locations, as other locations may have different characteristics. For instance, in developing countries or other locations where the use of WiFi are not common, the number of \ac{AP} in a public area may be much lower than what we have in our dataset.

As we mainly work with WiFi, the range of the proposed social density follows the maximum WiFi coverage for smartphones, which extends roughly from 20 to 50 meters. This range may be good or bad depending on the context. For outdoor situations, this range gives a good approximation as it is considerably broad enough to count social density. However, this range may be too broad for indoors. For instance, we may get a result that says there are 20 people in total, but in fact there is only five or even less people in the room. This is because WiFi also detects people outside the room. However, we can also combine the result with ambient noise recording, as empty room are usually more quite.

% Use wigle to explain that this could not be generalized. Mention comparison of one place and another place. in other cities or contries.

% #2: time
The other limitation is time. Our dataset consists of data collected in daytime, ranging from 08:15h to 14:45h. Using this dataset, we are only able to approximate the level of social density if the new data is captured during the same time frame. If the new data is taken outside the time frame, our dataset is unable to tell the level of social density. This fact is based on the time of scanning investigation described in~\autoref{ssub:effect_of_scanning_time} and \autoref{sub:effect_of_scanning_time}.


% conclusion
In conclusion, the results indicate that locations with high level of social density tend to have more access points. WiFi \ac{AP} shows the strongest correlation with the social density level, followed by \ac{RMS} that reveals a weaker correlation with social density level. Thus, we can say that it is possible to infer social density level from smartphone sensor readings. To achieve better accuracy, we may implement classification analysis instead of regression. However, regression analysis yields class-based result instead of continuous numerical result.

Furthermore, generalization of this method requires further investigation in more locations and time, as other location may reveal different patterns.
% Although the data say that there are minor variation in days, other locations may have higher variation.
Scanning time result also says that scanning time affects the social density level. Moreover, the settings and results will vary in different cities and even more in different countries.

%*****************************************
%*****************************************
%*****************************************
%*****************************************
%*****************************************

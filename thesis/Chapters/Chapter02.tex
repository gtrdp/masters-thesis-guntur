%!TEX root = ../thesis-guntur.tex
%*****************************************
\chapter{Related Work}\label{ch:related-work}
%*****************************************

Some researches have been done with regard to estimating social density, especially more are focused on crowd density.

Video processing has limitations such as weather conditions, illumination changes, limited viewing angle, and density and brightness problem. 

GSM location has an issue with privacy\cite{thesis017}.

MAC is only a proxy since it does not infer directly to personal information, such as name or contact.

A research from \cite{thesis008} proposes a way to detect crowds using Bluetooth. The crowd density is quantized into 7 groups, ranging from nearly empty to extremely high (crowded). Several features were also devised in this research, ranging from bla bla. 
The method was chosen due to bla bla.
The experiments were set up for 3 times, with 4 hours of duration each. 10 students were recruited to carry out the experiments.
The results show that bla bla.

Furthermore, \cite{thesis014} alleges that the existence of social relationships is possible to be uncovered by using WiFi probe signals.

Human queue is also possible to be monitored using WiFi, as demonstrated in \cite{thesis012}. It is based on RSSI that is measured by a single WiFi monitor.

WiFi and Bluetooth were also used to estimate crowd densities and pedestrian flows in \cite{thesis011}.

A research \cite{thesis017} utilizes MAC address data to determine spatio-temporal movement of human in terms of space utilization.

Bluetooth data is also used to analyze spatio-temporal movements of visitors event in Belgium \cite{thesis016}.


Also mention about noise.


[Why How]
Literature on Topic
Literature on Method
Theoretical Approach
Find a Hole
Look for debates


[Methodology]
Research design
Research procedures
Kind of data
collection procedures
selection and access
human subjects review
ethics statement
costs and funding

[Statement of Limitations]
Alternatives
weaknesses
what your research will do

[Conclusion]
Contributions
Importance
%*****************************************
%*****************************************
%*****************************************
%*****************************************
%*****************************************

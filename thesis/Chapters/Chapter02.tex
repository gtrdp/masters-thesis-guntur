%!TEX root = ../thesis-guntur.tex
%*****************************************
\chapter{State of the Art}\label{ch:literature-review}
%*****************************************
% This chapter would probably take around 10-20 pages

We overview several studies that have been done with regard to social density estimation. The methods vary from video monitoring, audio tone tracking, and \ac{RF} signal sensing. Furthermore, we present other applications of WiFi signal beyond crowd counting, such as pedestrian and queue monitoring. We also present studies related to ambulatory assessment that correlates with smartphone-based social density estimation. Finally, we present a brief discussion of the literature review that define our direction for the rest of the study.

% We also compare the approaches one another, focusing on the benefits, drawback, and challenges.
% create a summary or statement if the method suitable in our case or not
% We only address techniques which are possibly to be implemented in smartphone. Video or image based method, is not addressed because it is not suitable for smartphone, such as resource intensive~\cite{thesis034,thesis055}.

% Keyword used for literature study:
% crowd counting wifi
% crowd counting bluetooth
% wifi probe request
% online database social density
% counting crowd
% android monitor mode
% probe request wifi people
% crowd density wifi bluetooth
% social density online data
% crowd sensing wifi
% objective social density wifi bluetooth
% social density data
% social density objective
% speaker count -> filtered by year, >2012

\section{Crowd Counting Techniques} % (fold)
\label{sec:crowd_counting_techniques}
Several studies have been proposed to investigate possible technique to estimate the level social density. In this section, we summarize social density estimation techniques which use several approaches, such as WiFi, Bluetooth, other Radio Frequency (\ac{RF}) technology, video, and audio recording.
% GSM location has an issue with privacy\cite{thesis017}.

	\subsection{WiFi-based Techniques} % (fold)
	\label{sub:wifi-based-techniques}
	Within WiFi-based method, several approaches utilize one of the following, \ac{RSSI}, fingerprinting based, probe request, and \ac{CSI}.
	% We present each approach in separate subsections.

	\subsubsection{Received Signal Strength Indicator} % (fold)
	\label{ssub:rssi}
	Received Signal Strength Indicator (\ac{RSSI}) is one of the power measurement in a received radio signal. In WiFi, \ac{RSSI} indicates the power level received by the device after signal attenuation in the wireless transmission media. \ac{RSSI} is represented in negative value, where a closer value to zero indicates stronger received signal strength.

	Yoshida et al. presented a technique to infer the number of people present in a controlled environment~\cite{thesis052}. The authors use linear regression and support vector regression to model the number of people from collected dataset. The method is based on a fact that says the propagation of radio signal might be affected by people in the room, which affects the \ac{RSSI}. The method uses conventional WiFi \ac{AP} and computers to count people tested in laboratory experiments. Benchmarking with manual counting of people via camera as the ground truth, the method achieves 77.2\% of accuracy.

	Another proposed method to count people using \ac{RSSI} is proposed by Depatla et al.~\cite{thesis046}. Like the method proposed by Yoshida et al., it works based on the fact that says people leave traces in \ac{RF} signal, as they may block the line of sight propagation, resulting in a scattered radio signal, also known as multi-path phenomenon. The authors construct mathematical models of signal loss, which are based on the probability distribution of the received signal amplitude as a function of the total number of occupants, using Kullback-Leibler divergence. The method is tested both indoor and outdoor using a pair of pre-installed WiFi cards, involving nine people. The method achieves 96\% of accuracy for indoor and 63\% for outdoor, when using omni-directional WiFi antenna. 100\% of accuracy is also achievable using a directional antenna. Manual counting of people is used as ground truth information and the method is not tested for large crowds.

	% human queue
	Other than counting people, WiFi \ac{RSSI} can also be used to monitor human queues~\cite{thesis012}. The approach makes use of a single WiFi monitor located at the queue head (service desk), as opposed to conventional methods that rely on cameras or floor mats. The WiFi monitors collect signal traces indicated by \ac{RSSI}. As the \ac{RSSI} increases, the person is getting closer to the service desk. The queue phases are divided to three groups, namely wait, service period, and leave. A laboratory experiment with 90 traces of persons was conducted to test the method. Two-fold cross validation with manually logged ground truth are used to evaluate this method. Although the method seems promising, each queuing user is equipped with custom Android application that sends WiFi packets at 10 packets/second, which makes this method quite restrictive.

	% pedestrian flow
	Similar to queue monitoring, Schauer et al. proposed a method to estimate pedestrian flows and crowd densities~\cite{thesis011}. The authors propose several approaches, namely a na\"{i}ve approach, which works by only counting MAC addresses, a time-based approach, which also includes time information, a \ac{RSSI}-based approach, that includes WiFi signal strength, and a hybrid, which combines \ac{RSSI} and time. The authors use WiFi and Bluetooth in two identical and synchronized laptops to sense the signals. To test the method, the authors created single realistic scenario at a German airport in 16 days.
	Ground truth is provided by security check in German airports. The result shows that Pearson correlation is 0.75 on average in the hybrid approach and 0.93 for the best case. Although Bluetooth is mentioned, it is not significant in terms of results, as the Bluetooth-based method has a high number of false positive. 

	% movement of human
	Human movement pattern analysis is also a subject of study in a work by Abedi et al.~\cite{thesis017}. The method utilizes \ac{MAC} address data to determine spatio-temporal movement of humans in terms of space utilization. Specifically, this method leverages \ac{MAC} address in Bluetooth and WiFi. This method alleges that it could track group gathering and behavioral pattern. CrossCompass by Acyclica Inc. is used to capture \ac{MAC} address from both Bluetooth and WiFi. To prevent biased result, passing visitors, who visit the location below 4 minutes, are filtered. Groups of people are determined when several \ac{MAC} addresses enter and exit the lounge area in almost similar time. A \ac{MAC} address scanner is required to perform the measurement. The author conducted three weeks of data collection, resulting in data that consists of timestamp, \ac{MAC} address, and \ac{RSSI}, which is stored in 35,000 log lines.
	% The accuracy depends on how many devices are turned on.

	

	







	\subsubsection{WiFi Fingerprinting} % (fold)
	\label{ssub:wifi_fingerprinting}
	WiFi fingerprinting is a process of recording the \ac{RSSI} from several Access Points (\ac{AP}) in range and storing this data in a database together with the coordinates of the recording device. Later, any \ac{RSSI} vector at an unknown location is compared to those stored in the fingerprint database and the closest match is defined as the exact location. WiFi fingerprinting system may give an accuracy between 0.6 and 1.3 meters~\cite{Youssef2008}.

	Compared to the conventional \ac{RSSI} metering, WiFi fingerprinting requires prior knowledge of the monitored area. In the other way, a WiFi fingerprint map must be present to infer any \ac{RSSI} vector relative to a particular location.

	Kj{\ae}rgaard et al. proposed a method to count and track pedestrian flocks (groups) using WiFi fingerprinting~\cite{thesis033}. The technique uses a clustering method that operates in three different feature space: spatial (coordinates in controlled area), signal strength (within signal strength space), and pseudo-spatial (computed in a space with a defined distance unit). In the controlled environment, the author tested the method involving 16 subjects walking in a building at several floors, using manually annotated video recordings as the ground truth. The method provides 85\% F-measure accuracy. The authors allege that the accuracy may increase if the number of \ac{AP} increases.

	% plus PDR
	WiFi fingerprinting can also be combined with other techniques to monitor indoor crowds. A combination of WiFi fingerprinting and \ac{PDR} is proposed by Radu et al. to monitor indoor environment by means of crowd-sourcing~\cite{thesis020}. This method requires pre-deployed WiFi \ac{AP}s in the monitored area. This method, which works in indoor location, utilizes Ekahau mobile survey as the ground truth source.

	%>>>>> WiFi with combination with other instrument
	Another combination of WiFi fingerprinting is presented by Ahmet et al.~\cite{thesis045}. The study presents a method that combines geo-fencing with coarse WiFi fingerprinting for building evacuation. The method works on an assumption that says almost everyone carries a mobile phone and majority of the population have smartphones. Two major components of the method are magnetofence, a novel geo-fencing, which uses permanent magnets and phone magnetometer along with inertial sensors. Motion estimator is used to determine when to infer location and when to send data to the server. When a phone detects a fence, it locates the location and sends it to the server. The authors tested the method in an evacuation experiment that involves approximately 350 people. Using manual counting as the ground truth, the method achieves 98\% of accuracy in counting people getting in and out.


	\subsubsection{WiFi Probe Request} % (fold)
	\label{ssub:wifi_probe_request}
	Probe request is a data packet broadcast by WiFi-enabled devices, e.g., smartphones and tablets, to scan for available \ac{AP}. This mechanism is part of active scanning in WiFi standards~\cite{thesis082}, which is more energy efficient than passive scanning. Probe request packets contain detailed information about the corresponding device that broadcasts the packet, which makes it exploitable for people tracking.

	Schmidt showed that it is possible to infer social density estimation using captured WiFi probe request packets~\cite{thesis057}. The method works by capturing WiFi-packets from smartphones using dedicated scanning device: a Raspberry Pi with battery and D-Link WiFi module. The method notes device leaving or entering the monitored area. Database of \ac{MAC} address is created to filter out probe requests coming from any device other than a smartphone that may otherwise falsify the result. As a ground truth, four directions photos are taken in every 10 minutes. The method gives satisfying results but the author says that the method is not suitable for high-accuracy applications.
	
	Yaik et al. studied the correlation of crowd density with WiFi probe request~\cite{thesis047}. The study was performed in the university open day event that lasted for 8 hours, from 9.30AM to 4.30PM. As a ground truth checking, the authors used manual people counting using tally counter at the entrance of the event. A WiFi monitor, which scan WiFi probe request, is placed close to the entrance of the venue, nearby the tally counter volunteers. The study reveals 0.893 of correlation coefficient, which is considerably strong.

	Furthermore, Barbera et al. demonstrated that it is possible to infer the existence of social relationship other than merely crowd count~\cite{thesis014}. The method assumes that two or more users sharing one or more \ac{SSID} in their \ac{PNL} provide some information about the existence of social relationship between them. Affiliation network is used to construct the graph. The authors tested the method on a dataset collected in large national events, international events, city-wide probes, train station, and university, within three months of data collection. However, the author did not mention the accuracy of the proposed method.


	

	
	\subsubsection{WiFi Channel State Information} % (fold)
	\label{ssub:wifi_channel_state_information}
	Channel State Information (\ac{CSI}) points to known channel properties of a communication link that describe how a signal propagates and how the effects of, for instance, scattering, fading, and power decay, affect the communication. \ac{CSI} helps WiFi \ac{AP} to adapt the transmissions according to current channel conditions.
	% By investigating the effect of signal attenuation caused by human presence, we can model this relation to estimate the level of social density.

	%>>>> CSI
	Xi et al. presented that \ac{CSI}, which is sensitive to environment variation, has a monotonic relation with the number of moving people~\cite{thesis006}. The authors conducted experiments using multiple off-the-shelf IEEE 802.11n devices, organized in grid array, with up to 30 people in the area. Using manual counting of people as the ground truth, the authors implement Grey Verhulst model to model and train the data. Furthermore, the authors say that it is hard to get the ground truth when the number of people is large.

	In contrast with the study by Xi et al., Domenico et al. presented a method using \ac{CSI} without prior data training~\cite{thesis051}. The method works by implementing single WiFi transmitter and receiver from off-the-shelf devices. For less or equal than 2 person, the method achieves up to 91\% of accuracy. However, it is only possible to count up to 7 people and it is harder to perform the method especially if the WiFi transceiver (smartphone) is placed in the user's pocket.
	

	

	



	


	

	

	

	










	\subsection{Bluetooth-based Techniques} % (fold)
	\label{sub:bluetooth-based-techniques}
	Several works present studies of crowd counting techniques using Bluetooth. The techniques vary from individual to collaborative monitoring by using signal strength or Bluetooth \ac{MAC} addresses. 

	% Bluetooth - user behavior tracking, similar to our method, logging nearby bt devices
	Chen et al. proposed a method to track social contextual behavior using consumer smartphones~\cite{thesis028}. By using only Bluetooth traces captured in smartphones, the authors ran a data collection involving three volunteers in one to two weeks of experiments. A volunteer run the experiment during the day and fill out a questionnaire for ground truth checking at the end of the day. The proposed method uses six representative contextual behaviors for classification, namely working indoors, walking outdoors, taking subway, shopping in the mall, dining in the restaurant, and watching movie in cinema.
	C4.5 decision tree and 10-folds cross validation are utilized as evaluation method. The author alleges that the proposed method can achieve 85.8\% of accuracy. However, this method has very limited contextual behavior, which we think is not representative. Although a smartphone is utilized in this method, no explicit explanation about the crowd counting technique is present.

	% bluetooth - crowd count: collaborative
	As opposed to individual crowd counting, Weppner et al. proposed an approach of crowd counting using Bluetooth-based collaborative monitoring in smartphone~\cite{thesis008}. The crowd densities are quantized into 7 groups, ranging from nearly empty to extremely high (crowded), which will be the features in the training phase. The experiments were set up for 3 times, with 4 hours of duration each. The authors recruited 10 students to carry out the experiments, equipped with scanning smartphones that scans nearby Bluetooth signals. Using images taken by cameras as ground truth, the proposed method achieves 75\% accuracy.
	% Six features of scanning were developed to increase the accuracy of estimating the crowd.

	% also collaborative
	Another method of Bluetooth-based collaborative sensing achieved 82\% accuracy in the best case~\cite{thesis041}. The authors equipped several volunteers with Bluetooth enabled smartphone to walk through the crowd in three days analysis at Octoberfest, Munich, Germany. The estimation was carried out twice, individually and collaboratively. Features for individual estimation are the number of Bluetooth devices, the mean signal strength, and the variance of signal strength, while features for collaborative are average number of devices, variance of the number of device, and variance of all signal strength. The authors quantize the pictures depicting the ground truth into four levels, namely 0.1 people/m\textsuperscript{2}, 0.2 people/m\textsuperscript{2}, 0.3 people/m\textsuperscript{2}, and 0.4 people/m\textsuperscript{2}. The method gives 63\% for individual features and 81\% for collaborative features.

	% Bluetooth MAC address: by 22 bt scanners
	In contrast with smartphone based social density estimation method, Versichele et al. proposed a crowd counting method by employing 22 Bluetooth scanners placed around the monitored area~\cite{thesis016}. The proposed method analyzes spatio-temporal movements of visitors in Ghent Festivities, a 10 days event with approximately 1.5 millions of visitors. The scanners include a combination of class 1 (larger area) and class 2 (smaller area) Bluetooth scanner. The experiments yielded large datasets consisting of 260 millions of lines. To prevent counting the same person multiple times, the authors label a record with in, out, or pass. Moreover, to reduce bias from other Bluetooth devices, Bluetooth address are used to distinguish different types of Bluetooth device, e.g., phone or car handsfree, etc. Compared to the ground truth, which is provided by official visitor count, the method counts 11\% of the actual populations.
	

	









	\subsection{Other RF-based Techniques} % (fold)
	\label{sub:other_rf_techniques}
	\ac{RF} based techniques are not only limited to WiFi and Bluetooth, but also, for instance, \ac{WSN} and \ac{RFID} as described below.

	% RSSI and LQI: http://atmel.force.com/support/articles/en_US/FAQ/RSSI-vs-LQI-What-is-the-difference
	Nakatsuka et al. published a technique to estimate the crowd density using ZigBee\footnote{An IEEE 802.15.4-based specification for high-level communication by creating personal area networks for small and low-power digital wireless transmission.}~\cite{thesis042}. The proposed method exploits the ZigBee power measurement profile, namely \ac{RSSI} and \ac{LQI}, using a fact that says the signal strength is attenuated by human body as human body consists of approximately 60\% to 70\% of water, which is a medium that absorbs radio waves. The technique uses two ZigBee nodes, eight meters apart, for transmitting and receiving ZigBee signals. A model is constructed to count the number of people located between the transmitters. The method was tested using two scenarios, namely sitting and walking between nodes, which resulted in 300 samples.
	% However, this method is not suitable to be implemented in our study, as the technique requires pre-installed infrastructures.
	
	Another technique to estimate social density using \ac{WSN} was proposed by Yuan et al.~\cite{thesis043}. Contrasted with a techique from Nakatsuka et al. that employs at least two ZigBee devices, Yuan et al. employ tens of ZigBee devices.
	The authors conducted experiments in controlled situation of 16 grids (small scale) and 400 grids (large scale), in which sensor nodes were deployed in the point of intersection of the grids. The experiment yielded 168 samples (small scale) and 600 samples (large scale) of data. The result shows that \ac{RSSI} reading decreases as more people enter the grid. Then the variation signal strength is divided by the crowd density based to obtain the model. Using k-means clustering technique with manual counting as ground truth, the method can achieve 96\% accuracy.
	% Again, this WSN method is not suitable in our case, which requires no prior infrastructure.


	Another method works by utilizing \ac{RFID} tags. Mitchell et al. installed \ac{RFID} tags and readers across the city of Mecca, Saudi Arabia, to monitor pilgrims in Hajj~\cite{thesis050}. Basically, each pilgrim is given an \ac{RFID} tag, and optionally a smartphone application to locate and give them useful information or possibly sending an emergency signal. \ac{RFID} readers are placed in strategic locations around the Hajj area dividing it into several zones. For the future works, the authors plan to involve mobile or wireless provider cell tower to provide location and extend the service to buses or other service.









	\subsection{Other Crowd Counting Techniques} % (fold)
	\label{sub:other_crowd_counting_techniques}
	Vattapparamban proposed an approach of crowd counting and occupancy monitoring using an \ac{UAV} for outdoor monitoring~\cite{thesis053}. The proposed technique employs WiFi pineapple device as indoor monitoring mode. The approach works similarly by exploiting WiFi probe request transmitted by nearby devices. Having the WiFi probe request packets captured, the technique then applies \ac{k-NN} clustering and linear least square estimation to infer occupancy in a certain location. The author focuses on search and rescue applications. However, there are no validation checking and accuracy mentioned, as the author are more focused on occupancy tracking techniques. The study also describes \ac{MAC} address randomization observed in iOS 8, which is a potential threat for \ac{MAC} address based crowd counting technique.

	Audio tones are also tested to be a potential method to infer the level of social density~\cite{thesis044}. The approach uses the speakers and microphones available on consumer smartphones. The approach works by implementing tone exchange, in which each mobile device is installed with a tone counting application. A smartphone generates one or more simple tones and the other smartphones hear these unique patterns of sounds. The tone generating is done by using 2 main method: uniform hashing (involving a single tone) and geometric hashing (involving multiple tones). \ac{FFT} is used to analyze the received sound (peak analyzer), thus making background noise as one of the issues. Tested in three indoor and outdoor places with 25 Android phones, the proposed technique achieved up to 90\% accuracy. The author also alleges that the proposed approach is 81\% more energy efficient than WiFi-based approaches.

	The smartphone's microphone can also be utilized as a speaker counting instrument. The speaker counting technique estimates the number of people who participate in a conversation. Several works have been dedicated to this topic~\cite{thesis067,thesis074,thesis071}. Among those approaches, \ac{MFCC} is one of the effective coefficients for speech processing. The methods only work with four speakers or less.
	% However, in real situation, the speaker could be more than only four people.
	Furthermore, these techniques require \textit{a priori} information about the speakers, making it impossible to apply this method in our case. Another alternative of voice related technique, VAD might be a good candidate to detect whether there is a person speaking or not~\cite{thesis070}, but this technique only yields boolean value.
	% The authors also provide the source code for crowd detection\footnote{\url{https://github.com/lendlice/crowdpp}}.
	
	Ryan et al. proposed a method to count crowds using surveillance cameras~\cite{thesis034}. The method works by using tracking and local features to count the number of people in each group, so that the total crowd estimate is the sum of the group sizes. The authors use tracking to improve the robustness of the crowd count estimation, by analyzing the history of each group, including splitting and merging events. The study used manually annotated video recording as a ground truth. The method shows 1.6 mean average error. Video-based technique is also possible to be directly performed in the surveillance camera, equipped with low-cost single-board computer~\cite{thesis055}. The method achieves similar error as what has been proposed by Ryan et al., with mean average error of 1.68.



	





\section{Ambulatory Assessment} % (fold)
\label{sec:ambulatory_assessment}
Ambulatory Assessment is concerned with data capturing for decision making of patients treatment. Ambulatory assessment in everyday life is an emerging field, for instance, when dealing with patients suffering from neuropsychiatric disorders. Several works have addressed the ambulatory assessment topic, especially with mobile devices as the measuring instruments~\cite{thesis001,thesis031,thesis030,thesis015}.

A study presents an identification of relevant sensor sources in ambulatory assessment~\cite{thesis001}. The study implements an ambulatory assessment for context-aware \ac{ESM} in a mobile phone application. The authors conducted an online survey with ambulatory assessment experts to identify relevant sensor sources and contexts. The result alleges that time, date, and user activity, followed by location and notifications are prioritized smartphone sensors for context-aware experience sampling method. Tha authors also conducted a feasibility test to confirm that all mentioned relevant sources are available and accessible on Android devices.

Furthermore, Bachmann proposed smartphone-based social interaction sensing~\cite{thesis031}. The proposed technique monitors social interaction using WiFi, Bluetooth, audio, and user interactions in the smartphone. The technique uses nearby Bluetooth devices and devices within the same WiFi network as indicators of social interaction. Moreover, the notifications and subjects' activity in messaging applications are monitored.
The technique also implements unsupervised clustering method to infer the number of speaker in the patient's daily conversation, although it is not yet completed.
The technique is also monitoring the subjects location using \ac{GPS} and WiFi \ac{MAC} addresses. The logged data are aggregated in certain time intervals, eg., a week, for later analysis. Although the authors did not mention a reliable ground truth, they alleged that this proposal could achieve up to 76\% of accuracy for phone interaction.

% social well being
Another ambulatory assessment approach is proposed by Luo, et al.~\cite{thesis030}. The proposal puts attention on the daily multi-modal social activities done in smartphone, e.g., phone calls, text messages, and social network applications. Moreover, the technique also monitors phone books, GPS, WiFi, and Bluetooth proximity. The phone books are labeled on groups (friend, colleague, family, etc). In the other way, contact classification, place detection, proximity analysis, and application usage. Bluetooth is used only for detecting the type of nearby device, for instance, a salesman actively meets clients, while a software developer tends to stay in the office with the team members nearby.
% The experiment logged 106 mobile users, 107 days of data on average, 68,529 phone calls, 20,000 SMSes, and nearly 10k labeled contacts as the app saves log files, and periodically sent to server.














 






\section{Discussion} % (fold)
\label{sec:literature-discussion}
% intro
% \added{
	We present the summary of the literatures in~\autoref{tab:literature-summary}, based on the applications and their corresponding technologies. We categorize the instruments to gather the data into two categories, namely static and dynamic. Static instruments refer to instruments which are bound to a host that makes it relatively unmovable, while dynamic instruments refer to handheld devices which have certain physical dimensions and weight so that it is small enough to operate in the hand.
% }

% \added{
\begin{table}[h]
\begin{adjustwidth}{-2cm}{}
\centering
\caption{Literature review summary.}
\label{tab:literature-summary}
\begin{tabular}{llll}
\toprule
\multirow{2}{*}{Applications} & \multirow{2}{*}{Technologies} & \multicolumn{2}{c}{Instruments} \\ \cline{3-4}
                              &                               & Static         & Dynamic        \\ \midrule
Crowd counting         & WiFi (\ac{RSSI}) & \cite{,thesis046,thesis053,thesis012,thesis052} & \\
					   & WiFi (Fingerprint) & \cite{thesis045,thesis020}&                \\
					   & WiFi (Probe request) & \cite{thesis047}&\cite{thesis057}\\
					   & WiFi (\ac{CSI}) & \cite{thesis051,thesis006}&                \\
					   & Bluetooth & \cite{thesis016}&\cite{thesis041,thesis008}\\
					   & ZigBee & \cite{thesis042,thesis043} & \\
					   & Audio tone & & \cite{thesis044}\\
					   & Surveillance camera & \cite{thesis055,thesis034} & \\
Speaker counting       & Microphone & & \cite{thesis071,thesis074,thesis067}\\
Crowd surveillance     & RFID & \cite{thesis050} & \\
Voice activity detection & Microphone & & \cite{thesis070}\\
Pedestrian monitoring  & WiFi (\ac{RSSI}) & \cite{thesis017,thesis011}&                \\
					   & WiFi (Fingerprint) & \cite{thesis033}&                \\
\specialcellleft{Social relationship\\deduction} & WiFi (Probe request) & \cite{thesis014}& \\
\specialcellleft{Contextual behavior\\deduction} & Bluetooth & & \cite{thesis028} \\
Ambulatory assessment & Smartphone sensors &                & \cite{thesis031,thesis001,thesis015,thesis030}\\ \bottomrule
\end{tabular}
\end{adjustwidth}
\end{table}
% }

% \added{
	As we can see in~\autoref{tab:literature-summary}, approaches to crowd counting mostly involve static instruments, which are undesirable for passive behavioral monitoring as it requires static dedicated devices to monitor the patients. The ambulatory assessment techniques that works in a single smartphone do not address social density estimation topic explicitly. Several approaches that make use of dynamic instruments exist, which implement either WiFi (Probe request), Bluetooth, or audio tone.

	We plan to use a single smartphone as a monitoring tool of social density in the surroundings as it is able to work almost everywhere with no prior infrastructures. Based on~\autoref{tab:literature-summary}, we present a summary of available sensors in common modern smartphones in~\autoref{tab:smartphone-sensor-summary}, ranging from \ac{RF}-based sensor, such as \ac{GPS}, \ac{NFC}, and WiFi to motion sensor, such as gyroscope and accelerometer.
% }



\begin{table}[h]
\centering
\caption[Summary of typical smartphone sensors]
{Summary of typically available sensors in a modern smartphone. Sensors written in bold are preferred for social density estimation.}
\label{tab:smartphone-sensor-summary}
\begin{tabular}{lll}
\toprule
\multicolumn{3}{c}{Sensor Type} \\ \midrule
\textbf{Bluetooth}   & \ac{NFC}                    & gyroscope \\
\textbf{WiFi}        & cellular call               & proximity \\
\textbf{Microphone}  & touch screen                & compass \\
\textbf{Camera}      & fingerprint                 & barometer \\
                     & accelerometer               & \ac{GPS} \\
\bottomrule
\end{tabular}
\end{table}

% what we are going to do
In the present study, we would like to find a method to estimate the level of social density in the surrounding as a passive behavioral scheme. We make use of any sensors available on the smartphone, which can be used no matter how the smartphone is operated and positioned. We plan to use WiFi and microphone.

However, we do not use WiFi probe request based monitoring directly in the smartphone, as it is not possible due to restrictions in the mobile operating system. Probe request based estimation only works in WiFi monitor mode~\cite{thesis052,thesis079} and monitor mode is only accessible if the smartphone is rooted, jailbroken, or installed custom \ac{ROM}, which is considered as illegal in most countries~\cite{rootjailbreak}. We plan to note the number of available \ac{AP} in the surroundings using WiFi and record ambient noise using microphone.

% ground truth
To validate the result, we need to obtain the ground truth of the surroundings. Several methods have been proposed, ranging from manual counting using tally  counter to annotating video recording, but it is known that getting ground truth is difficult especially for large number of crowds~\cite{thesis006}. In the present study, we use two approximations of ground truth, namely image based method and WiFi probe request based method, as probe request is tested to have a strong correlation with the number of people nearby~\cite{thesis047}. However, recent mobile \ac{OS} has developed a technique named \ac{MAC} address randomization~\cite{thesis062,thesis061}. By using the randomization technique, the devices randomize their \ac{MAC} address in each particular circumstance to increase the privacy of the user and thus making the current people tracking technology no longer works~\cite{thesis079}.



%!TEX root = ../thesis-guntur.tex
%************************************************
\chapter{Introduction}\label{ch:introduction}
%************************************************

% The introduction chapter should contain:
% [What How Why]
% Research question
% Summary of proposal

% what is the general area you want to work in? why is it important?
% what is the state of the art, what have people been doing so far (some references)?
% why do you want to do something different and what? why is that new/cool/better/different/more elegant/more efficient/etc.?
% what is the problem you want to solve? why does it solve the issues/problems? with the previous approaches/methods/solutions and how does it do that?
% what is the methodology/steps that you want to use to approach the problem?
% what is the plan for the timeframe of your project?

% - In each chapter, write intro and conclusion explicitly. Preferably a paragraph for each.
% - Good intro is:
% 	- linking to prev. chapter (explicitly)
% 	- explaining the aim of this chapter.
% 	- explaining how to achieve the aim.
% - Better to use active voice, avoid passive voice as much as possible.
% - Good conclusion brings an intro to the next chapter.

% ==============================================================================
% How the intro should be
% - explain about the conventional method of personal well being, which 
%   has a problem in biased result, eg., are you always socializing? the person
%   answer yes, in fact, he is not.
%   also explain about socialization, the importance of social density
%   if possible, cite prof. kas publication on social density
% - also mention social well being app, which is good for objective measurement,
%   but no approach on social density, explain their approaches.
% - here the objective social count comes in, as it can provide an objective
%   measurement of social density. explain the approaches (including wifi, camera, etc)
%   that have been proposed
% - explain what it lacks of, such as it does not applicable to smartphone, must be rooted
% - explain where we try to come in. and more about randomization and VAD.
% - explain the research questions

% -> Explain something from psychological perspective.
% -> better to use some psychological literature
The current method of questionnaire is sometimes biased, thus objective one is needed. That is where crowd counting comes. Several works have addressed personal well-being applications to help this problem.

% -> some highlights might come from paul's thesis?
Sometimes, social density estimation is needed, especially in psychological monitoring of patients.

personal well being has been developed. However, none of those has presented a way to efficiently estimate social density.

Social density estimation, or sometimes referred as crowd counting, is the mechanism of estimating the number of people in certain area by means of a proxy, which could replace the manual counting method. Surveillance camera utilization is one of the method used, although it is limited by high deployment or computational cost. In recent years, some researches have proposed several method to estimate social density by means of RF signals, e.g., WiFi or Bluetooth, to get more low deployment or computational cost.

Other than mentioned above, the social density estimation may come handy in life. It could help to achieve certain tasks in many fields, for instance, crowd surveillance~\cite{thesis050}, evacuation and rescue~\cite{thesis045}, retail store customer analysis, even infrastructure development and evaluation, or queue management~\cite{thesis012}.

% -> introduction about probe request

% -> explanation about mac address randomization

% -> summary of this work
This study serves as 

% any relation of unique device and access point count?
To uncover the relation between the number of unique devices and available access points, I formulated the first research question:
\begin{displayquote}\textit{
Is there any correlation between number of unique devices and number of available Access Points in a certain area?}
\end{displayquote}

% how do the parameters affect?
When the data is captured, several parameters are also used, such as location, time, duration, and so on. Here comes the second:
\begin{displayquote}\textit{
How do the parameters affect the correlation result?}
\end{displayquote}

% how to resolve the randomization issue?
As mentioned previously, MAC address randomization is one of the challenges of estimating nearby unique device using probe-request. The third research question is somewhat related to that:
\begin{displayquote}\textit{
Is there any method to overcome the MAC address randomization issue? Which affecting number of unique devices.}
\end{displayquote}

% is VAD useful?
Furthermore, the validation using sound is also used and the last research question is asking about the effect of it.
\begin{displayquote}\textit{
How does validation using Voice Activity Detection help to achieve better result?}
\end{displayquote}

The rest of this thesis is structured as follows. Chapter \ref{ch:related-work} describes the related works, which are closely related to the social density estimation.


%*****************************************
%*****************************************
%*****************************************
%*****************************************
%*****************************************





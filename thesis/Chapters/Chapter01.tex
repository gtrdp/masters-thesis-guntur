%!TEX root = ../thesis-guntur.tex
%************************************************
\chapter{Introduction}\label{ch:introduction}
%************************************************

% - In each chapter, write intro and conclusion explicitly. Preferably a paragraph for each.
% - Good intro is:
% 	- linking to prev. chapter (explicitly)
% 	- explaining the aim of this chapter.
% 	- explaining how to achieve the aim.
% - Better to use active voice, avoid passive voice as much as possible.
% - Good conclusion brings an intro to the next chapter.

Social density estimation, or sometimes referred as crowd counting, is the mechanism of estimating the number of people in a particular area (social density) by means of a proxy, which could replace the manual counting method, e.g., counting heads by using tally counter. Social density estimation has many potential implementations. For instance, it can be implemented in mobile health care monitoring systems as a passive behavioral monitoring, in which the system monitors the level of social density that a patient encounters. The mobile health care would replace the conventional method to record a patient's social density investigation, which uses questionnaires, which sometimes causes the results to be biased.
Other implementations of social density estimation are, for instance, crowd surveillance~\cite{thesis050}, evacuation and rescue~\cite{thesis045}, retail store customer analysis, infrastructure development and evaluation, and queue management~\cite{thesis012}.

Several approaches of social density estimation exist. Surveillance camera utilization is one of the method used, although it is limited by high deployment or computational cost. Other than image based estimation, some researches have proposed Radio Frequency RF signals, e.g., WiFi or Bluetooth, mainly to get more low deployment or computational cost. The WiFi and Bluetooth based methods make use of several characteristics present in WiFi or Bluetooth, such as probe-request, MAC address monitoring~\cite{thesis008}, \ac{RSSI}~\cite{thesis046}, and \ac{CSI}~\cite{thesis051}.

Finding a method which is suitable and implementable as a passive behavioral monitoring is another problem. To the best of our knowledge, there is no approach of this.
The present study aims to implement social density estimation as a part of a passive behavioral monitoring using consumer smartphone as the measuring instrument. The result would help as to monitor patients . 
\cite{thesis015} % paul eskes
\cite{thesis031} % anja

To the best of our knowledge, no prior \cite{thesis001}
To the best of our knowledge, there is no approach that tries to combine or fusion several sensor measurements. Furthermore, no prior investigation of crowd counting using smartphone has been proposed, as mainly approaches are leveraging dedicated sensing devices.

explain that behapp is already counting available access point also cite
explain why we do this, say that probe request has a high correlation with the people count, cite the paper that says this
The result will be useful for BeHapp aplication [cite sociability]


also mention social well being app, which is good for objective measurement, but no approach on social density, explain their approaches.

the present study tries to figure out a solution that will work best especially for passive behavioral monitoring, where special requirements are present.

-> summary of this work

here the objective social count comes in, as it can provide an objective
measurement of social density. explain the approaches (including wifi, camera, etc) that have been proposed
explain what it lacks of, such as it does not applicable to smartphone, must be rooted

\section{Research Questions} % (fold)
\label{sec:research_questions}
This present study serves as an extension of BeHapp application. The present study focuses on the passive behavioral monitoring on smartphone.
BeHapp\footnote{\url{https://play.google.com/store/apps/details?id=org.behapp&hl=nl}}

This leads to the main research question:
\begin{displayquote}\textit{
How can we estimate the level of social density of the surroundings using smartphone as a passive behavioral monitoring scheme?}
\end{displayquote}

In order to answer this question the following sub-questions are formulated:
\begin{enumerate}
	\item \textit{What sensors are available at consumer smartphone and which of those are favorable for the social density estimation method?}\\
	Modern smartphones are equipped with several sensors to help users in their works, e.g., cameras, proximity sensor, finger print sensor, etc. In spite of the sensors have their own actual function, some sensors must be capable of giving such information for social density estimation. We are interested in investigating the sensors which are useful to achieve social density estimation.

	\item \textit{How can we validate the method, meaning, to get the ground truth or the approximation of the ground truth?}\\
	To validate whether the smartphone sensor data really does something related to the social density, we have to compare the smartphone sensor data with the actual social density level, or the ground truth. We look a method to obtain such ground truth or at least a strong estimation of the ground truth.

	\item \textit{How to overcome or minimize the challenges of this technique?}\\
	Knowing the approach to estimate the social density level using smartphone, some challenges or limitations may emerge. We also figure out the solutions of these problems so that at least the side effect can be minimized.

	\item \textit{Can this method work everywhere? What is the scope of this method?}\\
	We aim to devise a method that would work almost anywhere. Although this may be too ambitious, there must be a certain limit of the present study. We formally portray the scope of the final result.
\end{enumerate}





\section{Thesis Structure Overview} % (fold)
\label{sec:thesis_overview}
The rest of this thesis contains the following.
\autoref{ch:literature-review} summarizes the relevant works that correlated with the present study. The experimental setup and the its implementation regarding the social density estimation in consumer smartphone is described in~\autoref{ch:experimental-setup}, while the result of the experiments are depicted in~\autoref{ch:results}. \autoref{ch:regression-and-discussion}~presents the analysis of collected datasets to estimate or predict social density level based on smartphone sensors reading. The conclusion of this thesis and the corresponding future works are described in~\autoref{ch:conclusion-future-work}.

%*****************************************
%*****************************************
%*****************************************
%*****************************************
%*****************************************





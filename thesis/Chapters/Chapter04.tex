%!TEX root = ../thesis-guntur.tex
%************************************************
\chapter{Results}
\label{ch:results} % $\mathbb{ZNR}$
%************************************************
In the previous chapter, we present the bla bla
In this chapter

Explain about the data extraction.

Mention all experiment setup.
- randomization
- wifi
	- in days
	- in time

Display the final result, using table

Remember to explain the randomization result.

Ground truth: show an example of low density and high density crowd, comparing the 4 locations.

Interesting finding: nexus 5x immediately sends out probe request when wake up from sleep.

Randomization results indicate that 1 minute is the preferred time.

Randomization Experiment result:Experiment result:
The phone immediately sends out probe request with original mac when it wakes up from sleep. Usually 4 or 10. If the phone has woken up from a long sleep, more than 1 minutes. When the screen is locked.
The phone keeps sending out using the same first 3 byte of address. Brutally changes the last 3 octets.
The SN is close, but not really sequential.
iPad: also keeps sending out probe request when the display is on.
Take note the changes of mac address of LG Nexus 5X.
Every 10 secs (roughly) sends out 2 probe request, same manufacturer, different last 3 octets. Sequential or close SN.
But when it is stable, it sends out every roughly 60 secs, with each different mac (but same ). Count of probe beacon: 1 or 2, or even 3.
To prove it, I turned off the phone and there is no such mac address.
The mac address SN is close, but not sequential.
The pattern: original -> 10 sec -> 4 times random -> 60 secs.
The SN is restarted when the phone is active (not sleeping).
The randomized MAC address: da:a1:19. And LG is always using that same mac?
For the next experiment, please make sure that your phone WiFi is switched off.
When the screen is on, the phone does not send out randomized mac address.
Also test it using tcpdump.
When the phone restarted, it firstly sends out real mac address.
Create flowchart to filter out randomized mac.
Every burst of probe request, the address changes.
Even when WiFi is off, LG occasinally sends out probe request with original mac.

Experiment result iPad:
The iPad keeps sending out randomized mac even tough the screen is on.
Then the loading ion appears, the burst of probe request also captured.
Then iPad wakes from sleep, the mac address also changes. But still randomized.
When the screen is on, it keeps sending out probe request within 3, to 10 secs.
Take notes the randomized mac addess.
Take notes the setting of ipad prior and after sim card installation.
When I switched off the WiFi and on, the mac address changes. The SN is also restarted.
No difference, when the sim card is installed or not.
When ipad is connected to an AP (ad hoc, from Nexus), it sends out original mac address.
But the SN is always restarted. (Chance of solution)
If the iPad is in sleep mode, it sends out roughly every 2 minutes (135 secs) or even 4 minutes. and the SN is always restarted.
The SN is always resetted in each burst.
The pattern, 2x2 minutes, 4 minutes, then change MAC address

Signal strength:
% http://www.wireless-nets.com/resources/tutorials/define_SNR_values.html
more than 40dB SNR = Excellent signal (5 bars); always associated; lightening fast.
25dB to 40dB SNR = Very good signal (3 - 4 bars); always associated; very fast.
15dB to 25dB SNR = Low signal (2 bars); always associated; usually fast.
10dB - 15dB SNR = Very low signal (1 bar); mostly associated; mostly slow.
5dB to 10dB SNR = No signal; not associated; no go.

Explain by conclusion, not using whole data
effect of day
effect of location
effect of scanning time
put it in a whole picture
signal strength: WiFi and probe request

Working on decibels.
% https://support.biamp.com/General/Audio/Peak_vs_RMS_Meters
Important, writing: also address the microphone sensitivity.
% https://support.biamp.com/General/Audio/Microphone_sensitivity
Sound level decreases by 6dB with each doubling of distance from the source.
The sound is already using ambient noise reduction.
Those microphones are attuned to a specific (and rather narrow) range of sound intensity.

Writing: present the scatter data with colors.

Presenting: use the graph of prediction error to estimate the error.
also for cross validation.

Possible graph:
\begin{itemize}
	\item Number of removed mac address
	\item number of walking people
	\item available ap vs unique device
	\item count of manufacturers comparison
	\item final scatter data
	\item number of scanned probe per device.
	\item number of people entering and leaving the area.
	\item time vs count of unique devices
	\item time vs count of access points
	\item unique devices vs access points
	\item randomized mac addresses count
\end{itemize}

Question: Does weather affect WiFi performance?

Possible research: generate separate graph that explain one location with different time window: using scatter plot.

Explain the features on the data.

Make the datasets available online.

Devote a section only to show the mac address randomization result.

Explain the result in percentage as well, instead of manual count.

Writing:
1 minute is selected to minimize the effect of MAC address randomization, as well as according to our experiment, 1 minute is the interval of sending probe request in several manufacturers (it subjects to change)
assumption when it sees cars or buses
sound depends when the people speak out or not
the setting may vary in cities or even countries.
explain why dont we use channel hopping scanning.

Make sure that the data is available publicly, mention the url.


show the graph of Phone manufacturer and the graph of show the graph of scanned device (comparison between laptops, smartphones, and other)

Explain the density of each location in bar chart, comparing both head count and probe request count.

explain the size of data dump (images (MB) and text (lines))

% maybe use this
Use table of simulation about randomized mac address using different time window.

show the correlation graph per day per location, then combined per day, then combined all location all day

go for detail first, then combine all measurement using correlation graph matrix

an object may appear more than once in the manual camera: we used pattern recognition to avoid multiple counting.

other probe request from other device may be captured during mac address randomization experiment.

%*****************************************
%*****************************************
%*****************************************
%*****************************************
%*****************************************

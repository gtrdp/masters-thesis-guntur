%!TEX root = ../thesis-guntur.tex
%************************************************
\chapter{Conclusion and Future Work}
\label{ch:conclusion-future-work} % $\mathbb{ZNR}$
%************************************************
In this thesis, we present that it is possible to infer the level of social density in the surroundings. The main research question is:
mention the important message, that wifi could be use topredict roughly social density. but to generalize we must get more data.

recall the chapters to draw conclusion.

\begin{displayquote}\textit{
How can we estimate the level of social density of the surroundings using smartphone as a passive behavioral monitoring scheme?}
\end{displayquote}

\begin{enumerate}
	\item \textit{What sensors are available at consumer smartphone and which of those are favorable for the social density estimation method?}\\
	

	\item \textit{How can we validate the method, meaning, to get the ground truth or the approximation of the ground truth?}\\
	

	\item \textit{How to overcome or minimize the challenges of this technique?}\\
	

	\item \textit{Can this method work everywhere? What is the scope of this method?}\\
	Explain the limitation of the study.
	Explain that this method does not work in restricted environment, such as university location.
	An the other limitation.
\end{enumerate}

explain about this method could works as a localization method, could be useful especially if GPS signal is not available

\section{Future Works} % (fold)
\label{sec:future_works}
In the future work, we are interested in working more closely with estimating social density. Regression model based on the data to get the count of people. Machine learning method is also interesting to be implemented.

Future work, examine the correlation between ap count and location.
future work, examine the solutino of mac address randomization using probe request data.

add more feature, eg, time, for future works
add more data, eg, more locations with wider condition

devise more accurate modeling.

investigate other sensors to increase the accuracy of the method.

take care of outliers.

%*****************************************
%*****************************************
%*****************************************
%*****************************************
%*****************************************

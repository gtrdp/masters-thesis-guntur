%!TEX root = ../thesis-guntur.tex
%*******************************************************
% Abstract
%*******************************************************
%\renewcommand{\abstractname}{Abstract}
\pdfbookmark[1]{Abstract}{Abstract}
\begingroup
\let\clearpage\relax
\let\cleardoublepage\relax
\let\cleardoublepage\relax

\chapter*{Abstract}
Recent developments in smartphone technologies raise the concept of mobile healthcare systems as an essential part of medical care or research processes. As opposed to the conventional techniques which are prone to biased results and human errors, smartphone based monitoring systems can provide objective results especially when dealing with longitudinal assessment of individual movement patterns in the context of social density.\\

\noindent
In this thesis, we present a consumer smartphone based social density estimation method that estimates the number of people in a certain area by utilizing smartphone sensors. We use WiFi to count nearby Access Points (AP) and microphones to record ambient noise of the surroundings. We performed data collection in several locations, ranging from low to high level social densities, using WiFi MAC address counting and time-lapse images as the ground truth approximation.\\

\noindent
The results indicate that smartphones have good potential for estimating social density levels. The result reveal that the AP and ambient noise have a positive correlation with the social density level, which in our experiments is 0.8 and 0.6, respectively. Furthermore, we also constructed prediction models for the social density level using new data with residual error is equal to 7.05.
% As for the future work, we plan to enrich the smartphone data by the inclusion of other data types.

\endgroup			

\vfill
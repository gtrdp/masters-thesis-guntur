%!TEX root = ../thesis-guntur.tex
%*******************************************************
% Abstract
%*******************************************************
%\renewcommand{\abstractname}{Abstract}
\pdfbookmark[1]{Abstract}{Abstract}
\begingroup
\let\clearpage\relax
\let\cleardoublepage\relax
\let\cleardoublepage\relax

\chapter*{Abstract}
Recent developments of smartphone technologies raise the concept of mobile healthcare systems as an essential part of medical care or research processes. As opposed to the conventional techniques which are prone to biased results, smartphone based monitoring systems can provide objective result especially when dealing with longitudinal assessment of individual movement patterns in the context of social density.\\

\noindent
In this thesis, we present consumer smartphone based social density estimation method that estimates the number of people in a certain area by utilizing smartphone sensors. We use WiFi to count nearby Access Points (AP) and microphone to record ambient noise of the surroundings. We performed data collections in several locations, ranging from low to high level of social density, using WiFi MAC address counting and time-lapse images as the ground truth approximation.\\

\noindent
The results indicate that smartphone is potential for estimating social density level. The result reveal that the AP and ambient noise have a positive correlation with the social density level, which is 0.8 and 0.6, respectively. Furthermore, we also constructed prediction models to predict the social density level using new data with residual error is equal to 7.05. As for the future work, we plan to enrich the smartphone data by the inclusion of other data types.

\endgroup			

\vfill